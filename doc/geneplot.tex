\documentclass{article}

\begin{document}
\title{A Guide to the routines in the geneplotter package}
\maketitle

\section{Introduction}

The geneplotter package contains the graphics and visualiation
functionality for the Bioconductor project. 


\section{Microarray Images}

For most microarray experiments some variant of a plot that groups (or
orders) both samples and genes according to similar levels of
expression is made.
Typically one performs agglomerative hierarchical clustering to
determine which samples (or genes) are close to each other.
A dendrogram imposes some restriction on the order of the observations
but not a complete one. For a dendrogram based on $n$ observations
there are $2^{n-1}$ orderings that are consistent with the dendrogram
ordering. Thus a second, external quantity is used to select one of
these orders. Often that quantity is the average expression for either
the sample or gene (depending on which was being clustered).

There are many other potential means of ordering the observations.
One example would be to use the order derived from a minimal spanning
tree or from a single--linkage cluster analysis.
Carr and Olsen (1996) propose using the first principle component
scores as a means of ordering. They also suggest using single linkage
clustering and {\em to borrow the ordering from the ensuing
  dendrogram}.


In the genecluster package are functions for ordering observations
within clusters and ordering the clusters.


The basic building blocks for such a plot are the ability to reorder
both the rows and the columns of an array according to classifications
given by the clustering algorithm




\end{document}
